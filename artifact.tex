\section{Abstract}
\label{sec:abstract}
This artifact includes source code, scripts and datasets required to reproduce the experimental figures in the
evaluation of the MM'18 paper, which is entitled ``MiniView Layout for Bandwidth-Efficient 360-Degree Video''
\cite{Xiao:2018:MLB:3240508.3240705}. The artifact reports the comparison results among the standard cube layout
(CUBE), the equi-angular layout (EAC), and the MiniView layout (MVL) in terms of compressed video size, visual quality
of views and decoding and rendering time.


\section{Artifact description}
\label{sec:description}
The artifact is available at

{\footnotesize {\tt {\url https://github.com/bingsyslab/mm19-artifact}}}

\noindent without the datasets used in the evaluation. The datasets are accessible at

{\footnotesize {\tt {\url https://dl.dropboxusercontent.com/s/snlomfjoh7ybsk3/dataset.tgz}}}

\noindent, and it can also be automatically downloaded by running the script in the artifact. The artifact includes a
patch to {\tt ffmpeg} ({\tt 360-project} filter), running scripts, documentations and a small dataset for
verification.


\subsection{Folder structure}
\label{sec:folder}
The folder structure of the artifact is shown as follows:
%\begin{lstlisting}[language=bash, numbers=none, frame=none, morekeywords={mm19-artifact, layouts, shaders, Diving, 1280}]
\begin{lstlisting}[language=bash, numbers=none, frame=none]
  mm19-artifact/
  |- makefile
  |- makefile.sub
  |- cdf.pl
  |- cdf.gp
  |- config-output-pdf.gp
  |- remap.pl
  |- README.md
  |- 360_project.patch
  |- layouts/
     |- %.lt
  |- shaders/
     |- %.glsl
  |- Diving/
\end{lstlisting}
     %% |- 1280/
     %%    |- %.mp4
     %%    |- %.txt

\noindent{\tt makefile}: the script controlling experimental setups, evaluations, and plotting.

\noindent{\tt makefile.sub}: the script carrying out evaluations for one video.

\noindent{\tt cdf.pl}: the script collecting metrics from a set of videos.

\noindent{\tt config-output-pdf.gp}, {\tt cdf.gp}: the scripts plotting CDF figures.

\noindent{\tt remap.pl}: the script producing 360-degree videos with a specific layout from an HD equirectangular video.

\noindent{\tt README.md}: the documentation of the artifact.

\noindent{\tt 360\_project.patch}: the patch file to {\tt ffmpeg} that installs the {\tt 360- project} filter.

\noindent{\tt layouts/}: the folder holding different layout files.

\noindent{\tt shaders/}: the folder holding relevant OpenGL shaders.

\noindent{\tt Diving/}: the example video dataset.

\subsection{Datasets}
\label{sec:datasets}
The videos and user traces are from publicly available
datasets~\cite{Corbillon:2017:VHM:3083187.3083215,Wu:2017:DEU:3083187.3083210}. For fairness, we transcode all videos
into the equirectangular layout at the resolution of 3840x2048, and these videos are chunked into segments of 1
second. In addition, the user orientation traces are unified into {\tt .txt} files, in which a line is a sample and it looks like

{\tt 0.141033 4.000000 -15.787280 174.084166}

\noindent. The items are the timestamp in seconds, the line number, and the head rotations surrounding x-axis and
y-axis, respectively. We ignore the head rotation surrounding z-axis since users rarely change their head around this
axis. The video segments and the orientation files are named in the formats of {\tt \$\{sec\}\_sec.mp4} and {\tt
  uid-\$\{uid\}\_raw.txt}, respectively.


\begin{table*}\centering
\ra{1.2}
\caption{Dataset description}
\label{tab:dataset}
\begin{tabular}{@{}lll@{}} \toprule

   &  \# of traces & Video names \\
% \cmidrule{2-4} \cmidrule{6-8} \cmidrule{10-12}
\midrule
    Dataset-1~\cite{Corbillon:2017:VHM:3083187.3083215} & 53 & {\bf Diving}, Paris, {\bf Rollercoaster}, Timelapse, Venise \\
    Dataset-2~\cite{Wu:2017:DEU:3083187.3083210}        & 48 & Conan\_Gore\_Fly, Cooking\_Battle, {\bf Front}, Help, Rhinos, \\
                                                        &    & Conan\_Weird\_Al, Football, {\bf Tahiti\_Surf}, FemaleBasketball, \\
                                                        &    & Fighting, Korean, Reloaded, RioVR, TFBoy, VoiceToy,  Anitta  \\
\bottomrule
\end{tabular}
\end{table*}

The datasets include 21 videos. 5 of them have 58 users' view orientations while the other 16 videos have 48 users'
head movement traces. The list of these video names are shown in Table~\ref{tab:dataset}. In the evaluation, we
distinguish them into two groups: {\it moving-camera} and {\it static-camera} videos. In Table~\ref{tab:dataset}, the
{\it moving-camera} videos are presented in bold.


\section{Installation}
\label{sec:installation}
Before the experiment workflow, an {\tt ffmpeg} with {\tt 360-project} filter need to be built and all datasets should
be downloaded.

\subsection{Dependencies}
The {\tt 360-project} filter has been tested with the {\tt master} branch of {\tt ffmpeg} with the ref

{\tt 37e4c226c06c4ac6b8e3a0ccb2c0933397d6f96f}

\noindent and is expected to be correctly built with the latest {\tt ffmpeg} source code. The experiments are
successfully carried out on Ubuntu 18.04 and Ubuntu 16.04. The tested OpenGL version is OpenGL 3.3 (Core Profile) Mesa
18.0.5. The scripts in the artifact use Gnuplot 5.2 patchlevel 2, Perl 5.018 and GNU Make 4.1.

To install the software dependencies on a Ubuntu system, run the following commands:
\begin{lstlisting}[language=bash, numbers=none, morekeywords={sudo, apt,-,get, install}, deletekeywords={enable}]
$ sudo apt-get install autoconf automake build-essential \
  cmake git-core libass-dev libfreetype6-dev libsdl2-dev \
  libtool libva-dev libvdpau-dev libvorbis-dev libxcb1-dev \
  libxcb-shm0-dev libxcb-xfixes0-dev pkg-config \
  texinfo wget zlib1g-dev libglew-dev libglfw3-dev \
  libx264-dev gnuplot make yasm
\end{lstlisting}

\noindent{\bf Hardware dependencies:} Running this artifact on a platform equipped with a GPU is highly recommended,
which can greatly accelerate the projection operations.

\subsection{Running artifact over {\tt ssh}}
Since the artifact is developed based on GLFW, it requires a display to create a window, though it is invisible. For
running this artifact on a server  without a monitor connected via {\tt ssh}, the virtual framebuffer needs to be
installed:
\begin{lstlisting}[language=bash, numbers=none, morekeywords={sudo, apt, -, get, install}, deletekeywords={enable}]
$ sudo apt-get install xvfb
\end{lstlisting}
In the experiment workflow, two commands (``{\tt make remap-videos}'' and ``{\tt make view-videos}'') processing videos with
OpenGL need to be prefixed with {\tt svfb-run} to ensure their successful run. For example:
\begin{lstlisting}[language=bash, numbers=none, deletekeywords={enable},  morekeywords={make, git, clone, xvfb, -, run}]
$ xvfb-run make view-videos VNAMES=Diving
\end{lstlisting}
Details of these two commands are discussed in Section~\ref{sec:gen-videos}.

%% For running this artifact on a server via {\tt ssh} which has a monitor connected, use the following command to specify the
%% display name before further actions:
%% \begin{lstlisting}[language=bash, numbers=none, morekeywords={make, git, clone}, deletekeywords={enable}]
%% $ export DISPLAY=:0
%% \end{lstlisting}
%% In a rare case, the server display has another name, like ``{\tt :1}'' other than ``{\tt :0}''. This can be confirmed by
%% the following command:
%% \begin{lstlisting}[language=bash, numbers=none, morekeywords={make, git, clone}, deletekeywords={enable}]
%% $ who
%% username   :1       yyyy-mm-dd hh:mm (xxx.xxx.xxx.xx)
%% \end{lstlisting}
%% For successfully running this artifact on a server without a monitor connected, the output need to be forwarded to an
%% X-Window server. Different solutions need to be adopted for clients with different operating systems.

%% \noindent{\bf Windows}: we take {\tt VcXsrv} as the example for Windows clients. 1) Install {\tt VcXsrv} as the X-Window
%% server. 2) Use {\tt XLaunch} to launch the X-Window server and locate the window name at the left upper corner, such as
%% ``{\tt DESKTOP-9E5LKU7:0.0}''. 3) Use the following command to connect to the server
%% \begin{lstlisting}[language=bash, numbers=none, morekeywords={make, git, clone}, deletekeywords={enable}]
%% $ DISPLAY=DESKTOP-9E5LKU7:0.0 ssh -Y ${SERV_IP}
%% \end{lstlisting}



\subsection{Artifact installation}
\begin{enumerate}[leftmargin=*]
\item Download the artifact with the following instructions:
\begin{lstlisting}[language=bash, numbers=none,  morekeywords={make, git, clone}]
$ git clone https://github.com/bingsyslab/mm19-artifact
\end{lstlisting}
%% \end{enumerate}


\item Download, patch, and build {\tt ffmpeg} with {\tt 360-project} filter.
  \begin{itemize}[leftmargin=*]
  \item Use the {\tt makefile} script to automatically download, patch, and build {\tt ffmpeg}:
  \end{itemize}
\begin{lstlisting}[language=bash, numbers=none,  morekeywords={make, git, clone}]
$ cd mm19-artifact
$ make prepare-ffmpeg
\end{lstlisting}
If {\tt ffmpeg} has been successfully built, two binaries {\tt ffmpeg/ffmpeg} and {\tt ffmpeg/ffprobe} can be found.

%% \begin{itemize}[leftmargin=*]
\begin{itemize}[leftmargin=*]
\item If manually building {\tt ffmpeg} is preferred, follow the instructions below:
\end{itemize}
%% \end{itemize}
\begin{lstlisting}[language=bash, numbers=none, morekeywords={make, git, clone}, deletekeywords={enable}]
  $ cd mm19-artifact
  $ git clone https://github.com/FFmpeg/FFmpeg ffmpeg
  $ cd ffmpeg
  $ git apply ../360_project.patch
  $ ./configure --enable-gpl --enable-libx264 \
      --extra-libs='-lglfw -lGLEW -lGL -lGLU'
  $ make -j8 ffmpeg ffprobe
\end{lstlisting}

%% \begin{itemize}[leftmargin=*]
\item To download the datasets:
%% \end{itemize}
\begin{lstlisting}[language=bash, numbers=none, morekeywords={make, git, clone}, deletekeywords={enable}]
  $ make prepare-dataset
\end{lstlisting}
This will automatically download a tarball of the dataset named {\tt dataset.tgz} and extract it into a list of
folders. The tarball is 6.9G and the dataset takes additional 7.6G disk space after decompression. The tarball can also
be accessed via the link shown in Section~\ref{sec:description}. A video to be tested is organized in the format:
\begin{lstlisting}[language=bash, numbers=none, frame=none, morekeywords={mm19-artifact, layouts, shaders, Diving, 1280}]
  Diving/
   |- 1280/
      |- %.mp4
      |- %.txt
\end{lstlisting}
For a video ``Diving'', it should contain the HD equirectangular 360-degree video segments and the orientation
files. Both of them are stored in the directory {\tt 1280/}. Following the details in Section~\ref{sec:datasets} helps
add a new test video into the experiment.


\end{enumerate}


\section{Experiment workflow}
\label{sec:workflow}

The experiment follows three major steps: 1) 360-degree video generation, 2) view video generation, and 3) experimental measurements. A
full run of this artifact on the provided 21 videos (with the default configurations) takes $\sim$12 hours, and
additional $\sim$80G disk space will be occupied. The workflow can also be carried out on a single video for proving its
completeness. In this section, we will first present how to carry out the experiment on an example video
``Diving''. Then we have an integrated command to run the full experiment presented in our original paper. We will
further show how to customize this artifact.


\begin{table*}\centering
\ra{1.2}
\caption{Experimental parameters that can be customized}
\label{tab:customization}
\begin{tabular}{@{}llllr@{}} \toprule

  Name & Description & Value format & Default value & Value as a vector?\\
% \cmidrule{2-4} \cmidrule{6-8} \cmidrule{10-12}
\midrule
    {\tt VNAMES} & The folder names of tested videos & \{ {\tt Diving}, {\tt Paris}, $\ldots$\} & All videos & Yes \\
    {\tt MOVING\_VNAMES} & The names of videos as moving-camera type & \{ {\tt Diving}, {\tt Paris}, $\ldots$\} & {\tt Diving}, {\tt Rollercoaster}, & Yes \\
     &  &  & {\tt Tahiti\_Surf}, {\tt Front} &  \\
    {\tt LAYOUTS} & The names of tested layouts & \{{\tt cube}, {\tt eac}, {\tt mv}\} & {\tt cube eac mv} & Yes\\
    {\tt SCHEMES} & The compression parameters passed to {\tt x264} & \{{\tt crf\$\{N\}}\} & {\tt crf23} & Yes\\
    {\tt UID\_NR} & The number of users' orientation traces to test & {\tt \$\{N\}} & {\tt 20} & No \\
    {\tt TIMES\_NR} & The number of video segments to test & {\tt \$\{N\}} & {\tt 10} & No \\
    {\tt CUBE\_RES} & The resolution of CUBE layout & {\tt \$\{RES\}x\$\{RES\}} & {\tt 1920x1280} & No\\
    {\tt EAC\_RES} & The resolution of EAC layout & {\tt \$\{RES\}x\$\{RES\}} & {\tt 1920x1280} & No\\
    {\tt MV\_RES} & The resolution of MiniView layout & {\tt \$\{RES\}x\$\{RES\}} & {\tt 2240x832} & No\\
    {\tt FOV} & The FoV of generated view videos & {\tt \$\{DEGREE\}x\$\{DEGREE\}} & {\tt 100x100} & No\\
    {\tt VIEW\_RES} & The resolution of generated view videos & {\tt \$\{RES\}x\$\{RES\}} & {\tt 800x800} & No\\
\bottomrule
\end{tabular}
\end{table*}

\subsection{Video generation}
\label{sec:gen-videos}
\begin{enumerate}[leftmargin=*]
  \item To generate 360-degree videos for a given video:
\begin{lstlisting}[language=bash, numbers=none,  morekeywords={make, git, clone}]
$ make remap-videos VNAMES=Diving
\end{lstlisting}
    This command generates different layouts of 360-degree videos, where the option {\tt VNAMES} specifies which video
    is used. This artifact is extensively customizable and more details can be found in
    Section~\ref{sec:customization}. The output videos are generated into {\tt crf23/remaps/} in the names of {\tt
      \$\{layout\}\_\$\{time\}.mp4}.

  \item To generate view videos:
\begin{lstlisting}[language=bash, numbers=none,  morekeywords={make, git, clone}]
$ make view-videos VNAMES=Diving
\end{lstlisting}
The view videos directly generated from the HD equirectangular videos ({\tt 1280/\$\{time\}.mp4}) are named as {\tt
  \%.hd.mp4} in {\tt views/} while the view videos generated from different layouts of 360-degree videos are located at
{\tt crf23/views/}.
\end{enumerate}

\subsection{Video analysis}
\begin{enumerate}[leftmargin=*]
  \setcounter{enumi}{2}
\item To measure various metrics:
\begin{lstlisting}[language=bash, numbers=none,  morekeywords={make, git, clone}]
$ make psnr-logs VNAMES=Diving
$ make ssim-logs VNAMES=Diving
$ make ts-logs VNAMES=Diving
\end{lstlisting}
    These commands measure PSNR, SSIM as well as rendering and decoding time, which are preserved as log files stored at
    {\tt crf23/psnr/}, {\tt crf23/ssim/}, and {\tt crf23/ts/}, respectively. With the videos and log files generated, a
    test video folder looks like:
\begin{lstlisting}[language=bash, numbers=none, frame=none]
  |- Diving/
    |- 1280/
      |- ${time}.mp4
      |- uid-${uid}_raw.txt
    |- crf23/
      |- remaps/
        |- ${layout}_${time}.mp4
      |- views/
        |- ${time}.${uid}.${layout}.mp4
      |- psnr/
        |- ${time}.${uid}.${layout}.log
      |- ssim/
        |- ${time}.${uid}.${layout}.log
      |- ts/
        |- ${time}.${uid}.${layout}.log
    |- views/
      |- ${time}.${uid}.hd.mp4
\end{lstlisting}

\item Generate the figures in the PDF format:
\begin{lstlisting}[language=bash, numbers=none]
$ make cdfs pdfs VNAMES=Diving
\end{lstlisting}
Two types of files are generated with this command. {\tt \%.cdf} is the data file and {\tt \%.pdf} files are the
figures. There will be four figures showing the measured PSNR ({\tt psnr\_avg.crf23-psnr.pdf}), SSIM ({\tt
  All.crf23-ssim.pdf}), video size ({\tt size.crf23-remaps.pdf}) and rendering time ({\tt ts.crf23-ts.pdf}).

\end{enumerate}


\subsection{Rerun the full experiment}

To repeat the experiments presented in our original paper, use the following command
\begin{lstlisting}[language=bash, numbers=none]
  $ make psnr-ssim-ts-rand moving-static-cdfs moving-static-pdfs
\end{lstlisting}
The configuration of this experiment is the default ones shown in Section~\ref{sec:customization}. It worth noting that
this will generate 7 figures since we distinguish the 21 videos into {\it moving-camera} videos and {\it static-camera}
videos for PSNR, SSIM and video size measurement.


%% \item Generate 360-degree videos, view videos, PSNR logs, SSIM logs and decoding and rendering time logs with random
%%   selected video segments and user orientation traces:
%% \begin{lstlisting}[language=bash, numbers=none]
%%   $ make psnr-ssim-ts-rand
%% \end{lstlisting}
%% The experiments take $\sim$12 hours to complete, and additional $\sim$80G disk space is needed. After the experiments,
%% all videos and logs are generated under the corresponding video folders. A video folder looks like
%% \begin{lstlisting}[language=bash, numbers=none, frame=none, morekeywords={Diving, 1280, crf23, remaps, views, psnr, ssim, ts}]
%%   |- Diving/
%%      |- 1280/
%%         |- ${time}.mp4
%%         |- uid-${uid}_raw.txt
%%      |- crf23/
%%         |- remaps/
%%            |- ${layout}_${time}.mp4
%%         |- views/
%%            |- ${time}.${uid}.${layout}.mp4
%%         |- psnr/
%%            |- ${time}.${uid}.${layout}.log
%%         |- ssim/
%%            |- ${time}.${uid}.${layout}.log
%%         |- ts/
%%            |- ${time}.${uid}.${layout}.log
%%      |- views/
%%         |- ${time}.${uid}.hd.mp4
%% \end{lstlisting}
%% The 360-degree videos with different layouts are generated into {\tt crf23/remaps/}. The view videos directly generated from
%% the HD equirectangular videos ({\tt 1280/\$\{time\}.mp4}) are named as {\tt \%.hd.mp4} in {\tt views/} while the view
%% videos generated from different layouts of 360-degree videos are located at {\tt crf23/views/}. PSNR, SSIM, and
%% rendering and decoding time logs are stored at {\tt crf23/psnr/}, {\tt crf23/ssim/}, and {\tt crf23/ts/},
%% respectively.



\subsection{Experiment customization}
\label{sec:customization}
The experiment can be extensively customized. The comprehensive customizable parameters are listed in
Table~\ref{tab:customization}. An example of customized experiment can be launched like
\begin{lstlisting}[language=bash, numbers=none]
  $ make psnr-ssim-ts-rand VNAMES=Paris \
    FOV=90x90 LAYOUTS=mv SCHEMES=crf18 UID_NR=1 TIMES_NR=5
\end{lstlisting}
This means we only carry experiments over the {\tt Paris} video and only test the MiniView layout. When generating the
MiniView videos, the compression parameter uses {\tt crf=18}. The FoV of view videos is 90$^{\circ}$x90$^{\circ}$. We
randomly test over 5 video segments and 1 orientation trace.




\section{{\tt 360-project} filter}
\label{sec:filter}
We implement {\tt 360-project} as an {\tt ffmpeg} filter that generates a user view from an input 360-degree
frame. Understanding this tool is not necessary to evaluate the MiniView Layout. We will give a brief description to
this tool and further details can be found in the documentation of this artifact.

An example of using this filter is like:
\begin{lstlisting}[language=bash, numbers=none, morekeywords={make, git, clone}, deletekeywords={enable}, basicstyle=\ttfamily\tiny]
  $ ./ffmpeg -loglevel 'info' -y -i cube.mp4 -filter:v \
    "project=800:800:90:90:0:180:0:vertex.glsl:eqdis.glsl::cube.lt:0.0:1.0" \
    back.mp4
\end{lstlisting}
This command generates a view video from the 360-degree video {\tt cube.mp4}. The output view video has the resolution
of 800x800, and its FoV is 90$^{\circ}$x90$^{\circ}$. The video shows what a user will watch as she turns her head
following \{0$^{\circ}$, 180$^{\circ}$, 0$^{\circ}$\}, which is exactly the view from her back. An orientation file can
be passed in to simulate the user's head movement. {\tt vertex.glsl} and {\tt eqdis.glsl} are the vertex shader and the
fragment shader used to produce the view video. {\tt cube.lt} is the layout file describing the input video.


\section{Notes}
\label{sec:notes}
\begin{itemize}[leftmargin=*]
\item In the artifact, a list of commands can be used to ease clean operations:
\end{itemize}
\begin{lstlisting}[language=bash, numbers=none]
  $ make clean-remaps
  $ make clean-views
  $ make clean-logs
  $ make clean-cdfs
  $ make clean-pdfs
\end{lstlisting}
These commands remove remapped videos, view videos, experimental logs, CDF data files, and PDF figures,
respectively.

\begin{itemize}[leftmargin=*]
\item If you prefer building the {\tt ffmpeg} at another place, you need to redirect the {\tt ffmpeg} folder in the
  {\tt makefile}:
\end{itemize}
\begin{lstlisting}[language=bash, numbers=none]
  export FFMPEG_DIR := ${PATH_TO_FFMPEG}
\end{lstlisting}


%% \section{Artifact}

%% \subsection{Abstract}
%% This artifact includes source code, scripts and data set required to reproduce the experimental figures in the
%% evaluation of the MM'18 paper, which is entitled ``MiniView Layout for Bandwidth-Efficient 360-Degree Video''
%% \cite{Xiao:2018:MLB:3240508.3240705}. The artifact reports the comparison results among the standard cube layout
%% (CUBE), the equi-angular layout (EAC), and the MiniView layout (MVL) in terms of compressed video size, visual quality
%% of views and decoding and rendering time.

%% \subsection{Artifact check-list (meta-information)}
%% \begin{itemize}[leftmargin=*]
%%   \item {\bf Algorithm}: Standard Cube projection, Equi-angular projection, MiniView projection.
%%   \item {\bf Program}: A patched {\tt ffmpeg} executables.
%%   \item {\bf Compilation}: Use {\tt ffmpeg} building system.
%%   \item {\bf Binary}: {\tt ffmpeg} and {\tt ffprobe}.
%%   \item {\bf Data set}: publicly available equirectangular 360-degree videos (.mp4) and orientation files (.txt).
%%   \item {\bf Run-time environment}: ???
%%   \item {\bf Hardware}: GPU
%%   \item {\bf Metrics}: The video segment size, the per-segment average PSNRs and SSIMs, and the wall-clock time of decoding and rendering in different layouts.
%%   \item {\bf Output}: The experimental figures in PDF format, 360-degree videos in different layouts, and view videos.
%%   \item {\bf Experiment customization}: The FoV and the view video resolution can be altered. The encoding flag can be chosen from {\tt crf} or {\tt cbr}. The videos, the video segment number of each video and the user orientation trace number to evaluate can be changed.
%%   \item {\bf How much disk space required (approximately)?}: 70GB
%%   \item {\bf How much time is needed to complete experiments (approximately)?}: 10 hours
%%   \item {\bf Publicly available?}: Yes
%%   \item {\bf Code/data license (if publicly available)?}: GPL
%%   \item {\bf Archived?}: Yes
%% \end{itemize}

